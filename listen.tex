\chapter{Listen}
\label{cha:listen}
\section{Die Listenarten}
Es gibt eine Vielzahl von Paketen für die Unterstützung von Listen. Ich verwende lediglich das Pakte \verb*+enumitem+.

Es gibt drei unterschiedliche Arten von Listen, die als Standard in allen Dokumentklassen zur Verfügung stehen:
\begin{itemize}
	\item \verb*+itemize+ für Markierungslisten
	\item \verb*+enumerate+ für Aufzählungslisten
	\item \verb*+description+ für Schlagwortlisten	
\end{itemize}

Ein Beispiel für eine \emph{Markierungsliste} ist oben zu sehen. Die Verschachtelung wird bis in die vierte Ebene durchgeführt und jeweils mit einem anderen Symbol gekennzeichnet:
\begin{itemize}
	\item Erste Ebene
	\begin{itemize}
		\item Zweite Ebene
		\begin{itemize}
			\item Dritte Ebene
			\begin{itemize}
				\item Vierte Ebene
				\item Nochmals vierte Ebene
			\end{itemize}
			\item Immer noch dritte Ebene
		\end{itemize}
		\item Immer noch zweite Ebene
	\end{itemize}
	\item Immer noch erste Ebene
\end{itemize}

Die Symbole lassen sich durch \verb*+\renewcommand+ beliebig definieren\footnote{Anleitung dazu bei Voss, Wissenschaftliche Arbeit S. 130}.

Die \emph{Aufzählungsliste} funktioniert wie die Markierungsliste, nur mit dem Befehl \verb+enumerate+
\begin{enumerate}
	\item Erste Ebene
	\begin{enumerate}
		\item Zweite Ebene
		\begin{enumerate}
			\item Dritte Ebene
			\begin{enumerate}
				\item Vierte Ebene
				\item Nochmals vierte Ebene
			\end{enumerate}
			\item Immer noch dritte Ebene
		\end{enumerate}
		\item Immer noch zweite Ebene
	\end{enumerate}
	\item Immer noch erste Ebene
\end{enumerate}
Auch hier können Änderungen vorgenommen werden\footnote{Voss. S. 132 f.}.

Bei der \emph{Schlagwortliste} muss ein optionales Element verwendet werden, das standardmäßig fett gedruckt wird.
\begin{description}
	\item[AT] Altes Testament
	\item[NT] Neues Testament
\end{description}
Bei Verschachtelungen, die ebenfalls möglich sind, muss bedacht werden, dass nach jedem \verb*+\item+ ein Zeichen folgen muss, damit die nächste Ebene mit einer neuen Zeile begonnen wird (z. B. die Tilde). Ein Sonderfall der Schlagwortlioste ist die \emph{Labeling-Liste}\footnote{Weiteres siehe bei Voss. S. 133}.

\section{Das Paket enumitem}

