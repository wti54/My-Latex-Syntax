\chapter{Textelemente}
\label{cha:textelemente}
\section{Allgemeines}
\section{Zeilenabstand}
\section{Titelei}
\section{Seitenstil}
\section{Textausrichtung}
\section{Textauszeichnung}
\section{Hoch- bzw. tiefgestellter Text}
\section{Textboxen}
Frame oder fbox
\section{Fußnoten}
\section{Endnoten}
\section{Randbemerkungen}
\section{Textmarken und -referenzen}
\section{Hyperlinks}
\subsection{Allgemeines}
Das Paket \verb*+hyperref+ stammt von Sebastian Rahtz und Heiko Oberdiek. Es klinkt sich in sämtliche Referenzen, Links, Fußnoten usw. ein. \verb*+hyperref+ soll als letztes aller Pakete geladen werden. Allerdings gibt es eine Fehlermeldung, wenn das Paket \verb*+cleveref+ zuvor geladen wird. Die internen Links werden umrahmt dargestellt. Sie werden allerdings nicht ausgedruckt. Mit der Option \verb*+linkcolor=+ kann die Textfarbe eingestellt werden. Dabei verschwinden die Rahmen. Mit dem Makro \verb*+\url{URL}+ geschriebene Links werden automatisch umgebrochen. Beispiel URL: \url{http://www.dante.de}.
\subsection{Lesezeichen - Bookmarks}
Normalerweise werden Lesezeichen automatisch von \verb*+hyperref+ verwaltet. Man kann allerdings auch manuell eingreifen.
\subsection{Sonstiges}
Betrifft Anker, Farben und Links auf Gleitumgebungen.
\section{Textumbruch}
\section{Horizontale und vertikale Abstände}
\section{Mehrspaltiger Text}
\section{Kritische Editionen}
\section{Große Dokumente}



